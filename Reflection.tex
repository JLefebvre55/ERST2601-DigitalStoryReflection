\documentclass{report}
\usepackage{setspace} % Setting line spacing
\usepackage{ulem} % Underline
\usepackage{caption} % Captioning figures
\usepackage{subcaption} % Subfigures
\usepackage{geometry} % Page layout
\usepackage{multicol} % Columned pages
\usepackage{array,etoolbox}
\usepackage{fancyhdr}
\usepackage{enumitem}
\usepackage[table]{xcolor}
\usepackage[toc,page]{appendix}
\usepackage{titlesec} % Section formatting

\usepackage[backend=biber,style=apa,citestyle=authoryear]{biblatex}
\DeclareLanguageMapping{english}{english-apa}
\DeclareFieldFormat{journaltitle}{\textit{#1}}
\DeclareFieldFormat[article]{volume}{\textit{#1}}
\DeclareFieldFormat[misc]{title}{\textit{#1}}
\DeclareFieldFormat[inbook]{booktitle}{\textit{#1}}
\addbibresource{references.bib}

\titleformat{\section}{\normalfont\fontsize{12}{15}\bfseries}{\thesection}{1em}{}
\titleformat{\section}{\normalfont\fontsize{12}{15}\bfseries}{\thesection}{1em} % space between number and text
  {} % formatting for just the text

% Page layout (margins, size, line spacing)
\geometry{letterpaper, left=1in, right=1in, bottom=1in, top=1in}
\setstretch{2}

% Headers
\pagestyle{fancy}
\lhead{ERST2601 Critical Reflection}
\rhead{Jayden Lefebvre}

\begin{document}

\begin{titlepage}
    \begin{center}
        \vspace*{1.2cm}

        \textbf{Storytelling for Engineering: Application of Relationality and Biocultural Frameworks of Indigenous Environmental Knowledge Systems in Systematic Problem-Solving}

        \vspace{1.25cm}

        Jayden Lefebvre\\

        \vspace{5cm}
        
        Trent University\\
        ERST 2601Y 2024WI\\
        Dr. Dan Longboat\\

        \vfill

        Word Count: \textbf{2000}\\
        February 10th, 2025
        
    \end{center}
\end{titlepage}

\clearpage

\section{Who Was I?}

My background in engineering has presented a significant hurdle when trying to ground my learning in Indigenous spirituality and knowledge systems: stakeholdership vs relationality, design (utilitarian function, form/formed) praxis vs biocultural (sovereign land, living/lived) praxis, systems thinking vs land-people-culture thinking, and the application of subjective lived experience and storytelling to problem-solving.

Ancestral intelligence in design and relationality in solution-stakeholdership \footnote{To explore and begin to put into practice the \textbf{interactions of Indigenous and Western science in critical decision-making} to create solutions to common environmental issues.}, consideration of perspectives and values, and placing your focus where it matters most; Personal growth and resetting priorities, the importance of personal relationships and community, and the significance of solving ``people problems'' over technical feats; Why does creating/solving it matter enough to do it and make it a part of who I become?

\section{What Have I Learned?}

My re-understanding human beings and myself as being inseparable from the natural world \footnote{To understand that Indigenous Environmental Knowledges are a \textbf{way of being}, and \textbf{inseparable from aspects of daily living} such aesthetics, spirituality, metaphysics, social order, ceremony and most importantly the land.}, rediscovery of a love for the environment and transfer to Trent, the importance of the land and sources of Indigenous knowledge, and what I've learned so far:

Organic, regenerative practices, agroecology (SAFS); sustainability, ecosystem services, resource extraction (ERSC); Climate Change, indicator species, biodiversity (BIOL); the Biocultural Framework, and the significance of diverse sources of truth in constructing a worldview for problem-solving/engineering in the post-modern age (IESS).

Through this learning I have come to understand that there exists a necessity for a unifying narrative framework of being for all peoples that can be passed on from person to person through generations; one that is instrinsically humanist and ecologically sound and that is grounded in a sense of self-sovereignty of the Earth, its resources, and all living things \footnote{To explore and begin to understand Indigenous land/water rights and responsibilities, \textbf{rights to self-determination and self-government}, and the diversity of strategies Indigenous Peoples use to resist colonization and globalization and the importance of cultural revitalization in terms of environmental and land protection.}. Without a treaty for each, each is indeed doomed.

\section{Who Am I Now?}

Treaties tell stories of the land, our agreements, and the formation of new friendships. In the same way, permits and regulations on resource extraction (in Indigenous contexts) must take into account historical relationships between sovereign peoples and the land they steward; in the same way that individuals require treaties to liberate themselves from institutional tyranny, the ``bioculture'' needs regulation to liberate itself from individual tyranny and the exploitation of ecosystem services via unsustainable practices. An evolution of storytelling for the post-modern age would regard our time in the present day as a winter (e.g. material, spiritual) whose purpose is to remind us to love what we had - so that we may preserve and protect what we may well have once again. In this way we will achieve ample preparation for a truly uncertain future \footnote{To investigate \textbf{revitalization, resistance and collaborative strategies} used by Indigenous Peoples in reaction to the land and natural environment. }.

Wildlife dissapearance due to trophic collapse and loss of biodiversity serve as indicator species of climate change, and have led to the loss of traditional food and medicine sources of Indigenous peoples, particularly those that inhabit areas without access to arable farmland \parencite{jonassangris}. How we go about fixing such a systemic problem directly relates to what stories we tell ourselves: therein lies the necessity for an Indigenous apocalyptic narrative (à-la-Revelations - the Horsemen have come and gone; what now?). Plant trees and the caribou shall return \parencite{nwtcaribou}; ``Oui - et ca devient possible!''

The youth are the future, but the elders are the present; they set the stage for what is to come. If we are to be successful, we must be aware and capable, and be led by those who have come before. We have seen what happens when that order is inverted: those who are the most aware/conscientious are the most likely to ``drop out'' and turn to apathy and nihilism; those who are the most capable/intelligent are the most likely to ``sell out'' and turn to frivolity and hedonism.

If we are to engineer a future for seven generations to come, it must not be the burden of the young - yet, feeling the pressure, we turn consistent bombardment of information, leading to ``analysis paralysis''. The inversion of social expectation (elders looking to youth for guidance) is the first step in the shredding of the social treaty between the individual and the institution. Ask not what youth can do for you, but what you can do to empower the young.

\clearpage

% References
\printbibliography

\end{document}